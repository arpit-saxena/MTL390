\documentclass[12pt, oneside]{article}
\usepackage{a4wide}
\usepackage{oldgerm}
\usepackage{amsmath}
\usepackage{amssymb}
\usepackage{amstext}
\usepackage{booktabs}
\setlength{\textheight}{8.875in} \setlength{\textwidth}{6.875in}
\setlength{\columnsep}{0.3125in} \setlength{\topmargin}{0in}
\setlength{\headheight}{0in} \setlength{\headsep}{0in}
\setlength{\parindent}{1pc} \setlength{\oddsidemargin}{-.304in}
\setlength{\evensidemargin}{-.304in}

\newcommand{\argmin}{\operatornamewithlimits{argmin}}

\begin{document}
\setlength{\textheight}{8.5in}
\centering {\bf MTL 390 (Statistical Methods) }\\


\centering{\bf Minor Examination Assignment 1 Report}



\vskip 0.5cm

\noindent Name: Arpit Saxena ~  ~~~~~ ~~~~ ~~~~~~~~~~~~~~~~ Entry Number: 2018MT10742 ~~~~~~~~~~~~~



\vskip 0.5cm



\begin{enumerate}
	



\item	Descriptive Statistics


\item	Descriptive Statistics


\item	Sampling Distributions



\item	Sampling Distributions 


\item	{
  Point and Interval Estimations

  For a distribution with \(k\) unknown parameters, method of moments uses \(k\) moments
  to form a system of equations and solves it to find estimates for the parameters.
  This throws away information contained in higher order moments. To remedy that, the
  \textbf{Generalized Method of Moments (GMM)} takes \(q (> k)\) moments and minimizes
  the sum of squares of difference between sample moments and moments calculated from
  the distribution.

  Consider the following samples taken from a Poisson distribution with unknown \(\lambda\).
  Find an estimate for the parameter using both method of moments as well as generalized
  method of moments (with 3 moments)

  \begin{center}
  \begin{tabular}{cccc}
    \toprule
    30 & 21 & 24 & 18 \\
    28 & 25 & 24 & 25 \\
    26 & 19 & 19 & 21 \\
    22 & 34 & 22 & 15 \\
    22 & 25 & 16 & 22 \\
    \bottomrule
  \end{tabular}
  \end{center}

  \textbf{Answer}

  We first calculate expressions of three moments \(E[X], E[X^2] \text{ and } E[X^3]\) for
  \(X \sim P(\lambda)\).

  Using the MGF of the Poisson distribution, we find moments around 0:
  \begin{align*}
    M_X(t) &= exp(\lambda(e^t - 1)) \\
    \implies M_X'(t) &= \lambda e^t exp(\lambda e^t - \lambda) \\
    \implies M_X''(t) &= (\lambda e^t)^2 exp(\lambda e^t - \lambda) + \lambda e^t exp(\lambda e^t - \lambda) \\
    \implies M_X'''(t) &= (\lambda e^t)^3 exp(\lambda e^t - \lambda) + 2 (\lambda e^t)^2 exp(\lambda e^t - \lambda) \\
        &+ (\lambda e^t)^2 exp(\lambda e^t - \lambda) + \lambda e^t exp(\lambda e^t - \lambda) \\
        &= (\lambda e^t)^3 exp(\lambda e^t - \lambda) + 3 (\lambda e^t)^2 exp(\lambda e^t - \lambda) + \lambda e^t exp(\lambda e^t - \lambda) \\
  \end{align*}

  Using these, we calculate the moments as:
  \begin{align*}
    E[X] &= M_X'(0) = \lambda \\
    E[X^2] &= M_X''(0) = \lambda^2 + \lambda \\
    E[X^3] &= M_X'''(0) = \lambda^3 + 3\lambda^2 + \lambda
  \end{align*}

  Next we calculate the sample moments. Let the samples be written as \(x_1, ..., x_{20}\). Then:
  \begin{align*}
    m_1 &= \sum_{i=1}^{20} x_i  = 22.9\\
    m_2 &= \sum_{i=1}^{20} {x_i}^2 = 544.4 \\
    m_3 &= \sum_{i=1}^{20} {x_i}^3 = 13424.5 \\
  \end{align*}

  Using the method of moments, we get:
  \[\widehat{\lambda}_1 = m_1 = 22.9\]

  For the generalized method of moments, we note that we can take any weighting of the
  sample moments. In fact we can also take a positive definite matrix and define the
  cost function that way. Suppose we somehow decided to keep the weights for \(m_1, m_2, m_3\)
  to be \(100, 10, 1\) respectively. Then, we have the function:

  \begin{align*}
    Q(\lambda) &= 100(m_1 - E[X])^2 + 10(m_2 - E[X^2])^2 + 1(m_3 - E[X^3])^2 \\
               &= 100(22.9 - \lambda)^2 + 10(544.4 - \lambda^2 - \lambda)^2 + (13424.5 - \lambda^3 - 3\lambda^2 - \lambda)^2
  \end{align*}

  Then the estimator is given by:
  \[\widehat{\lambda}_2 = \argmin_\lambda{Q(\lambda)}\]

  We note that \(Q(\lambda)\) is a polynomial in lambda with degree 6, so it's not practical
  to calculate the minimum by hand. Using computer tools, we find:

  \[\widehat{\lambda}_2 = 22.79\]
}


\item	Point and Interval Estimations


\end{enumerate}

\end{document}
