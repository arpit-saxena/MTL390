\documentclass[12pt, oneside]{article}
\usepackage{a4wide}
\usepackage{oldgerm}
\usepackage{amsmath}
\usepackage{amssymb}
\usepackage{amstext}
\usepackage{cancel}
\setlength{\textheight}{8.875in} \setlength{\textwidth}{6.875in}
\setlength{\columnsep}{0.3125in} \setlength{\topmargin}{0in}
\setlength{\headheight}{0in} \setlength{\headsep}{0in}
\setlength{\parindent}{1pc} \setlength{\oddsidemargin}{-.304in}
\setlength{\evensidemargin}{-.304in}

\newcommand\numberthis{\addtocounter{equation}{1}\tag{\theequation}}

\begin{document}
\setlength{\textheight}{8.5in}
\centering {\bf MTL 390 (Statistical Methods) }\\


\centering{\bf Major Examination Assignment 2 Report}



\vskip 0.5cm

\noindent Name: Arpit Saxena ~~~  ~~~~~ ~~~~ ~~~~~~~~~~~~~~~~ Entry Number: 2018MT10742 ~~~~~~~~~~~



\vskip 0.5cm



\begin{enumerate}
	



\item	Testing of Hypothesis


\item	 Testing of Hypothesis


\item	Analysis of correlation and regression 



\item	Analysis of correlation and regression 


\item	Time Series Analysis


\item {
    Time Series Analysis

    Consider the stationary ARMA(p, q) process. Elaborate the general method of finding the
    variance and covariances. Use the method to find the variance and covariances \&
    correlations of time difference upto 3 (i.e. \(\rho_1, \rho_2, \rho_3\)) of the stationary ARMA(1, 1)
    process.

    \textbf{Answer}

    Let \(\left\{X_t\right\}_{t \in \mathbb{N}}\) be a time series following the
    stationary ARMA(p, q) process. Then we have the following recurrence relation:

    \[X_t = \sum_{j=1}^{p}\alpha_j X_{t-j} + \sum_{j=1}^{q} \beta_j e_{t-j} + e_t\]

    where \(e_t\)'s are independent and identically distributed random variables following
    a normal distribution with mean 0 and variance 1. We also take the boundary condition as
    \(X_t = 0 ~\forall t < max(p, q)\)

    Since this is a stationary process, it is implied that it will also satisfy
    wide sense stationarity, which means that
    \begin{align*}
        E(X_t) &= \mu ~\forall t \\
        Cov(X_t, X_s) &= f(t - s)
    \end{align*}

    i.e. the mean of all the random variables of the process is constant and
    the covariance of the random variables at two time instances of the process
    depends only on the time difference between them.

    We first find the mean of the random variables of the process:
    \begin{align*}
        X_t &= \sum_{j=1}^{p}\alpha_j X_{t-j} + \sum_{j=1}^{q} \beta_j e_{t-j} + e_t \\
        \implies E(X_t) &= E\left(\sum_{j=1}^{p}\alpha_j X_{t-j} + \sum_{j=1}^{q} \beta_j e_{t-j} + e_t\right) \\
        \implies E(X_t) &= \sum_{j=1}^{p}\alpha_j E(X_{t-j}) + \sum_{j=1}^{q} \beta_j E(e_{t-j}) + E(e_t)
        \intertext{Now we use the fact that \(e_t\)'s have a mean of 0 to get}
        E(X_t) &= \sum_{j=1}^{p}\alpha_j E(X_{t-j}) \\
        \implies \mu &= \sum_{j=1}^{p}\alpha_j \mu \tag*{(Since the process is stationary)} \\
        \implies \left(\sum_{j=1}^{p}\alpha_j - 1\right) \mu &= 0 \\
        \intertext{Now assuming that \(\sum_{j=1}^{p}\alpha_j \neq 0\), we get}
        \mu &= 0
    \end{align*}

    Therefore we have the following result:
    \begin{equation}
        E(X_t) = 0 ~\forall t \label{eq:q6:meanzero}
    \end{equation}

    We denote \(\gamma_\tau\) as the covariance of random variables in this process
    at time \(\tau\) apart. Then,
    \begin{align*}
        X_t &= \sum_{j=1}^{p}\alpha_j X_{t-j} + \sum_{j=1}^{q} \beta_j e_{t-j} + e_t \\
        \intertext{Multiplying by \(X_{t-\tau}\) on both sides, we get}
        X_t X_{t-\tau} &= \sum_{j=1}^{p}\alpha_j X_{t-j} X_{t-\tau} + \sum_{j=1}^{q} \beta_j e_{t-j} X_{t-\tau} + e_t X_{t-\tau} \\
        \implies E\left(X_t X_{t-\tau}\right) &= E\left(\sum_{j=1}^{p}\alpha_j X_{t-j} X_{t-\tau} + \sum_{j=1}^{q} \beta_j e_{t-j} X_{t-\tau} + e_t X_{t-\tau}\right) \\
        \implies E\left(X_t X_{t-\tau}\right) &= \sum_{j=1}^{p}\alpha_j E\left(X_{t-j} X_{t-\tau}\right) + \sum_{j=1}^{q} \beta_j E\left(e_{t-j} X_{t-\tau}\right) + E\left(e_t X_{t-\tau}\right) \tag*{(Since expectation is a linear operator)} \\
        \intertext{Using \eqref{eq:q6:meanzero}, we have \(Cov(X_t, X_s) = E(X_t X_s) - E(X_t)E(X_s) = E(X_t X_s)\)}
        \implies Cov\left(X_t, X_{t-\tau}\right) &= \sum_{j=1}^{p}\alpha_j Cov\left(X_{t-j}, X_{t-\tau}\right) + \sum_{j=1}^{q} \beta_j E\left(e_{t-j} X_{t-\tau}\right) + E\left(e_t X_{t-\tau}\right) \\
        \implies \gamma_{\tau} &= \sum_{j=1}^{p}\alpha_j \gamma_{\tau - j} + \sum_{j=1}^{q} \beta_j E\left(e_{t-j} X_{t-\tau}\right) + E\left(e_t X_{t-\tau}\right)
    \end{align*}

    Now we note that \(X_{t-\tau}\) is a function of the white noise variables \(e_1,\ldots,e_{t-\tau}\) and since
    they are independent from each other, \(e_t\) is independent from all of \(e_1,\ldots,e_{t-\tau}\) and thus \(E(e_t X_{t-\tau}) = Cov(e_t, X_{t-\tau}) + \cancelto{0}{E(e_t)} E(X_{t-\tau}) = Cov(e_t, X_{t-\tau}) = 0\).
    Therefore, we get
    \begin{equation}
        \gamma_{\tau} = \sum_{j=1}^{p}\alpha_j \gamma_{\tau - j} + \sum_{j=1}^{q} \beta_j E\left(e_{t-j} X_{t-\tau}\right) \label{eq:q6:arma_var}
    \end{equation}

    Using the previous logic, we can see that \(E(e_t X_s) = 0\) whenever \(t > s\) i.e. \(e_t\) and \(X_s\) are
    independent whenever \(t > s\). In lieu of this observation, we split into the following two cases:

    \renewcommand{\labelitemi}{\textendash}
    \begin{itemize}
        \item \textbf{Case 1: } \(\tau > q\)
        
        Here, \(j < \tau \implies t - j > t - \tau\) for all \(j = 1, \ldots, q\). This implies, by our previous observation,
        that \(E(e_{t-j} X_{t-\tau}) = 0\) for all \(j = 1,\ldots,q\)

        Thus the equation simplifies to:
        \begin{equation}
            \gamma_{\tau} = \sum_{j=1}^{p}\alpha_j \gamma_{\tau - j} \label{eq:q6:case1}
        \end{equation}

        \item \textbf{Case 2: } \(0 < \tau \leq q\)
        
        In this case, we split up the summation into two parts as:
        \begin{align*}
            \sum_{j=1}^{q} \beta_j E(e_{t-j} X_{t-\tau}) &= \cancelto{0}{\sum_{j=1}^{\tau - 1} \beta_j E(e_{t-j} X_{t-\tau})} + \sum_{j=\tau}^{q} \beta_j E(e_{t-j} X_{t-\tau}) \\
                &= \sum_{j=\tau}^{q} \beta_j E(e_{t-j} X_{t-\tau})
        \end{align*}

        For each of the terms we'll need to expand \(X_{t-\tau}\) to find the coefficient
        of \(e_{t-j}\) in it, which will give us the expectation.

        Thus we get the equation as:
        \begin{equation}
            \gamma_{\tau} = \sum_{j=1}^{p}\alpha_j \gamma_{\tau - j} + \sum_{j=\tau}^{q} \beta_j E(e_{t-j} X_{t-\tau})
                \label{eq:q6:case2}
        \end{equation}
    \end{itemize}

    Now, we find the variance of \(X_t\).
    \begin{align*}
        X_t &= \sum_{j=1}^{p}\alpha_j X_{t-j} + \sum_{j=1}^{q} \beta_j e_{t-j} + e_t \\
        \intertext{Multiplying by \(X_t\) on both sides, we get}
        X_t^2 &= \sum_{j=1}^{p}\alpha_j X_{t-j} X_t + \sum_{j=1}^{q} \beta_j e_{t-j} X_t + e_t X_t \\
        \implies E(X_t^2) &= \sum_{j=1}^{p}\alpha_j E(X_{t-j} X_t) + \sum_{j=1}^{q} \beta_j E(e_{t-j} X_t) + E(e_t X_t) \\
        \implies Var(X_t) &= \sum_{j=1}^{p}\alpha_j \gamma_j + \sum_{j=1}^{q} \beta_j E(e_{t-j} X_t) + E\left(e_t \left\{\sum_{j=1}^{p}\alpha_j X_{t-j} + \sum_{j=1}^{q} \beta_j e_{t-j} + e_t\right\}\right) \\
        \implies Var(X_t) &= \sum_{j=1}^{p}\alpha_j \gamma_j + \sum_{j=1}^{q} \beta_j E(e_{t-j} X_t) + \sum_{j=1}^{p}\alpha_j \cancelto{0}{E(e_t X_{t-j})} + \sum_{j=1}^{q} \beta_j \cancelto{0}{E(e_t e_{t-j})} + E(e_t^2)
    \end{align*}
    
    Therefore, we have the variance equation as:
    \begin{equation}
        Var(X_t) = \sum_{j=1}^{p}\alpha_j \gamma_j + \sum_{j=1}^{q} \beta_j E(e_{t-j} X_t) + 1 \label{eq:q6:var}
    \end{equation}

    Now we consider a stationary ARMA(1, 1) process with the equation
    \[X_t = \alpha X_{t-1} + \beta e_{t-1} + e_t\]

    To find the variance, we use \eqref{eq:q6:var}.
    \begin{align*}
        Var(X_t) &= \sum_{j=1}^{p}\alpha_j \gamma_j + \sum_{j=1}^{q} \beta_j E(e_{t-j} X_t) + 1 \\
            &= \alpha \gamma_1 + \beta E(e_{t-1} X_t) + 1 \\
            &= \alpha \gamma_1 + \beta E\left(e_{t-1} \left\{\alpha X_{t-1} + \beta e_{t-1} + e_t\right\} \right) + 1 \\
            &= \alpha \gamma_1 + \alpha \beta E(e_{t-1} X_{t-1}) + \beta^2 E(e_{t-1}^2) + \beta \cancelto{0}{E(e_{t-1} e_t)} + 1
    \end{align*}

    Now using \(E(e_{t-1}^2 = Var(e_{t-1}) = 1)\) and \(E(e_t X_t) = 1 ~\forall t\), we get
    \begin{equation}
        Var(X_t) = \alpha \gamma_1 + \alpha \beta + \beta^2 + 1 \label{eq:q6:var_calc}
    \end{equation}

    Note we have yet to find \(\gamma_1\) which we'll do next. Note that since \(0 < \tau = 1 \leq q = 1\),
    we'll use \eqref{eq:q6:case2} to calculate.
    \begin{align*}
        \gamma_{\tau} &= \sum_{j=1}^{p}\alpha_j \gamma_{\tau - j} + \sum_{j=\tau}^{q} \beta_j E(e_{t-j} X_{t-\tau}) \\
        \implies \gamma_1 &= \alpha \gamma_0 + \beta E(e_{t-1} X_{t-1}) \\
        \implies \gamma_1 &= \alpha \gamma_0 + \beta \\
        \implies \gamma_1 &= \alpha Var(X_t) + \beta \numberthis \label{eq:q6:gamma1_var} \\
        \intertext{Now using \eqref{eq:q6:var_calc}, we get}
        \gamma_1 &= \alpha (\alpha \gamma_1 + \alpha \beta + \beta^2 + 1) + \beta \\
        \implies \gamma_1 &= \alpha^2 \gamma_1 + \alpha^2 \beta + \alpha \beta^2 + \alpha + \beta \\
        \implies (1 - \alpha^2) \gamma_1 &= \alpha^2 \beta + \alpha \beta^2 + \alpha + \beta \\
        \implies \gamma_1 &= \frac{\alpha^2 \beta + \alpha \beta^2 + \alpha + \beta}{1 - \alpha^2} 
            \numberthis \label{eq:q6:gamma1_ans}
    \end{align*}

    Now that we have found \(\gamma_1\), we'll describe the other results using it since
    they get very messy otherwise. From \eqref{eq:q6:gamma1_var}, we have
    \begin{align*}
        \gamma_1 &= \alpha Var(X_t) + \beta \\
        \implies \gamma_1 &= \alpha \gamma_0 + \beta \\
        \intertext{Now dividing both sides by \(\gamma_0\) and setting \(\frac{\gamma_1}{\gamma_0}\) to \(\rho_1\), we get}
        \rho_1 &= \alpha + \frac{\beta}{\gamma_0} \\
        \implies \rho_1 &= \alpha + \frac{\beta}{ \alpha \gamma_1 + \alpha \beta + \beta^2 + 1} \tag*{(Using \eqref{eq:q6:var_calc})}
    \end{align*}

    For \(\tau > 1 = q\), we can use the equation for case 1 i.e. \eqref{eq:q6:case1}. We have
    \begin{align*}
        \gamma_{\tau} &= \sum_{j=1}^{p}\alpha_j \gamma_{\tau - j} \\
        \implies \gamma_\tau &= \alpha \gamma_{\tau-1} \\
        \intertext{Now dividing both sides by the variance to the correlations,}
        \rho_\tau &= \alpha \rho_{\tau-1} \\
        \therefore~ \rho_2 &= \alpha \rho_1 = \alpha^2 + \frac{\alpha \beta}{ \alpha \gamma_1 + \alpha \beta + \beta^2 + 1} \\
        \rho_3 &= \alpha \rho_2 = \alpha^3 + \frac{\alpha^2 \beta}{ \alpha \gamma_1 + \alpha \beta + \beta^2 + 1}
    \end{align*}

    Therefore, we have found the values of \(\rho_1, \rho_2, \rho_3\) as
    \begin{align*}
        \rho_1 &= \alpha + \frac{\beta}{ \alpha \gamma_1 + \alpha \beta + \beta^2 + 1} \\
        \rho_2 &= \alpha^2 + \frac{\alpha \beta}{ \alpha \gamma_1 + \alpha \beta + \beta^2 + 1} \\
        \rho_3 &= \alpha^3 + \frac{\alpha^2 \beta}{ \alpha \gamma_1 + \alpha \beta + \beta^2 + 1}
        \intertext{where, }
        \gamma_1 &= \frac{\alpha^2 \beta + \alpha \beta^2 + \alpha + \beta}{1 - \alpha^2}
    \end{align*}
}	


\end{enumerate}

\end{document}
